\documentclass{article}
\usepackage[margin=1in]{geometry}
\usepackage{amsmath}
\title{LP Formulation for Scheduler}

\date{\today}
\begin{document}

\maketitle

\section{Without Application Packing}
Given applications $1, 2, \ldots, n$, our goal is to make placement
decisions, assuming that the resources available is heterogeneous. For now, we
assume we have $m$ different resources: these $m$ resources can be of different types (for
example, servers with different generations of GPUs like K80s, P100s, and V100s).

We assume that a profiling run gives us rough performance estimates for each
of these applications (for example, we can time 100 minibatches of each application
on each hardware, or use collaborative filtering to extrapolate these
times from other measurements). The performance estimate for application $i$
on machine $j$ is represented by $a_{ij}$.

Now, let $x_{ij}$ be a floating point number, representing the fraction of
total wall-clock time spent executing the computation associated with
application $i$ on machine $j$.

\begin{eqnarray}
0 \leq x_{ij} \leq 1 & \forall (i,j) \in \{1, 2, \ldots, n\} \times \{1, 2, \ldots, m\} \nonumber  \\
\sum_j x_{ij} = 1 & \forall i \in \{1, 2, \ldots, n\} \nonumber
\end{eqnarray}

Now, the effective throughput observed by application $i$ is,
\begin{eqnarray}
y_i = \sum_j x_{ij} \cdot a_{ij} \nonumber
\end{eqnarray}

Now, we can maximize $\min_i y_i$ to find the allocation of applications
among different resources that maximizes the minimum throughput (that is,
finds a max-min allocation) -- the goal is to find an allocation that encourages
sharing (that is, an allocation that is better than just giving each user / application
$1/n$ of the cluster).

\section{With Application Packing}
The above formulation doesn't consider concurrent execution of applications
on any given machine. We can relax this restriction.

Instead of having a variable $x_{ij}$ for all combinations of application $i$
with machine $j$, we instead create a variable $x_{ijc}$,
which represents the fraction of epochs of application $i$ run on machine $j$
as part of application combination $c$.
Similarly, the performance estimate for an application $i$ when run as part of
a combination $c$ on machine $j$ is represented as $a_{ijc}$.

\begin{eqnarray}
0 \leq x_{ijc} \leq 1 & \forall i,j,c \nonumber \\
\sum_{c \in C_i} \sum_j x_{ijc} = 1 & \forall i \in \{1, 2, \ldots, n\} \nonumber
\end{eqnarray}

Here, $C_i$ represent all application combinations containing the application $i$.

We now define $y_i$ (effective throughput of application $i$) as,
\begin{eqnarray}
y_i = \sum_{c \in C_i} \sum_j x_{ijc} \cdot a_{ijc} & \forall i \in \{1, 2, \ldots, n\} \nonumber
\end{eqnarray}

Note that this formulation uses a total number of variables
linear in the total number of application combinations
under consideration. Can we do better? (probably)

Now, we can maximize $\min_i y_i$ to find the allocation of applications among
different resources that maximizes the minimum throughput (that is, finds a
max-min allocation).

\end{document}
